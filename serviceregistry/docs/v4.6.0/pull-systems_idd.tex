\documentclass[a4paper]{arrowhead}

\usepackage[yyyymmdd]{datetime}
\usepackage{etoolbox}
\usepackage[utf8]{inputenc}
\usepackage{multirow}

\renewcommand{\dateseparator}{-}

\setlength{\parskip}{1em}

\newcommand{\fparam}[1]{\textit{\textcolor{ArrowheadBlue}{#1}}}

%% Special references
\newcommand{\fref}[1]{{\textcolor{ArrowheadBlue}{\hyperref[sec:functions:#1]{#1}}}}
\newcommand{\mref}[1]{{\textcolor{ArrowheadPurple}{\hyperref[sec:model:#1]{#1}}}}
\newcommand{\pdef}[1]{{\textcolor{ArrowheadGrey}{#1 \label{sec:model:primitives:#1} \label{sec:model:primitives:#1s}}}}
\newcommand{\pref}[1]{{\textcolor{ArrowheadGrey}{\hyperref[sec:model:primitives:#1]{#1}}}}

\newrobustcmd\fsubsection[5]{
  \addtocounter{subsection}{1}
  \addcontentsline{toc}{subsection}{\protect\numberline{\thesubsection}operation \textcolor{ArrowheadBlue}{#1}}
  \renewcommand*{\do}[1]{\rref{##1},\ }
  \subsection*{
    \thesubsection\quad
    #2 \textcolor{ArrowheadPurple}{#3} \\
    \small
    \hspace*{0.075\textwidth}\begin{minipage}{0.1\textwidth}
      \vspace*{1mm}
      Operation: \\
      \notblank{#4}{Input: \\}{}
      \notblank{#5}{Output: \\}{}
    \end{minipage}
    \begin{minipage}{0.825\textwidth}
      \vspace*{1mm}
      \textcolor{ArrowheadBlue}{#1} \\
      \notblank{#4}{\mref{#4} \\}{}
      \notblank{#5}{\mref{#5} \\}{}
    \end{minipage}
  }
  \label{sec:functions:#1}
}
\newrobustcmd\msubsection[2]{
  \addtocounter{subsection}{1}
  \addcontentsline{toc}{subsection}{\protect\numberline{\thesubsection}#1 \textcolor{ArrowheadPurple}{#2}}
  \subsection*{\thesubsection\quad#1 \textcolor{ArrowheadPurple}{#2}}
  \label{sec:model:#2} \label{sec:model:#2s}
}
\newrobustcmd\msubsubsection[3]{
  \addtocounter{subsubsection}{1}
  \addcontentsline{toc}{subsubsection}{\protect\numberline{\thesubsubsection}#1 \textcolor{ArrowheadPurple}{#2}}
  \subsubsection*{\thesubsubsection\quad#1 \textcolor{ArrowheadPurple}{#2}}
  \label{sec:model:#2} \label{sec:model:#2s}
}
%%

\begin{document}

%% Arrowhead Document Properties
\ArrowheadTitle{pull-systems HTTP/TLS/JSON} %e.g. ServiceDiscovery HTTP/TLS/JSON
\ArrowheadServiceID{pull-systems} % e.g. register
\ArrowheadType{Interface Design Description}
\ArrowheadTypeShort{IDD}
\ArrowheadVersion{4.6.0}
\ArrowheadDate{\today}
\ArrowheadAuthor{Tamás Bordi} % e.g Szvetlin Tanyi}
\ArrowheadStatus{RELEASE}
\ArrowheadContact{tbordi@aitia.ai} % jerker.delsing@arrowhead.eu
\ArrowheadFooter{\href{www.arrowhead.eu}{www.arrowhead.eu}}
\ArrowheadSetup
%%

%% Front Page
\begin{center}
  \vspace*{1cm}
  \huge{\arrowtitle}

  \vspace*{0.2cm}
  \LARGE{\arrowtype}
  \vspace*{1cm}
\end{center}

%  \Large{Service ID: \textit{"\arrowid"}}
  \vspace*{\fill}

  % Front Page Image
  %\includegraphics{figures/TODO}

  \vspace*{1cm}
  \vspace*{\fill}

  % Front Page Abstract
  \begin{abstract}
    This document describes a HTTP protocol with TLS payload
    security and JSON payload encoding variant of the \textbf{pull-systems} service.
  \end{abstract}
  \vspace*{1cm}

\newpage

%% Table of Contents
\tableofcontents
\newpage
%%

\section{Overview}
\label{sec:overview}

This document describes the \textbf{pull-systems} service interface, which enables the systems to get data about all system registrated in the Local Cloud. It's implemented using protocol, encoding as stated in the following table:

\begin{table}[ht!]
  \centering
  \begin{tabular}{|l|l|l|l|}
    \rowcolor{gray!33} Profile type & Type & Version \\ \hline
    Transfer protocol & HTTP & 1.1 \\ \hline
    Data encryption & TLS & 1.3 \\ \hline
    Encoding & JSON & RFC 8259 \cite{rfc8259} \\ \hline
    Compression & N/A & - \\ \hline
  \end{tabular}
  \caption{Communication and semantics details used for the \textbf{pull-systems}
    service interface}
  \label{tab:comunication_semantics_profile}
\end{table}

This document provides the Interface Design Description IDD to the \textit{pull-systems -- Service Description} document.
For further details about how this service is meant to be used, please consult that document.

The rest of this document describes how to realize the \textbf{pull-systems} service HTTP/TLS/JSON interface in details.

\newpage

\section{Interface Description}
\label{sec:functions}

The service responses with the status code \texttt{200
  Ok} if called successfully. The error codes are, \texttt{400
  Bad Request} if request is malformed, \texttt{401 Unauthorized} if
improper client side certificate is provided, \texttt{500 Internal
  Server Error} if Service Registry is unavailable.

\begin{lstlisting}[language=http,label={lst:register},caption={A \fref{pull-systems} invocation.}]
GET /serviceregistry/pull-systems?direction={direction}&page={page}&item_per_page={size}&sort_field={sortField} HTTP/1.1
\end{lstlisting}

\begin{lstlisting}[language=http,label={lst:register_response},caption={A \fref{pull-systems} response.}]
{
  "data": [
    {
       "id": 0,
       "systemName": "string",
       "address": "string",
       "port": 0,
       "authenticationInfo": "string",
       "metadata": {
            "location": "building-a"
        },
       "createdAt": "string",
       "updatedAt": "string"
     }
  ],
  "count": 0
}
\end{lstlisting}

\newpage

\section{Data Models}
\label{sec:model}

Here, all data objects that can be part of the service calls associated with this service are listed in alphabetic order.
Note that each subsection, which describes one type of object, begins with the \textit{struct} keyword, which is meant to denote a JSON \pref{Object} that must contain certain fields, or names, with values conforming to explicitly named types.
As a complement to the primary types defined in this section, there is also a list of secondary types in Section \ref{sec:model:primitives}, which are used to represent things like hashes, identifiers and texts.

\msubsection{struct}{QueryParams}
\label{sec:model:QueryParams}

\begin{table}[ht!]
\begin{tabularx}{\textwidth}{| p{3cm} | p{3cm} | p{2cm} | X |} \hline
\rowcolor{gray!33} Field & Type & Mandatory & Description \\ \hline
direction & \pref{Direction} & no & Sorting direction. \\ \hline
page & \pref{Number} & no & Pagination page number. \\ \hline
size & \pref{Number} & no & Pagination page size. \\ \hline
sortField &\pref{String} & no & Field name used as the basis of the sorting. \\ \hline
\end{tabularx}
\end{table}

\msubsection{struct}{SystemListResponse}
\label{sec:model:SystemListResponse}

\begin{table}[ht!]
\begin{tabularx}{\textwidth}{| p{4.25cm} | p{3.5cm} | X |} \hline
\rowcolor{gray!33} Field & Type      & Description \\ \hline
data & \pref{List}$<$\hyperref[sec:model:SystemRecord]{SystemRecord}$>$     & List of service instances. \\ \hline
count & \pref{Number} & Size of the result list. \\ \hline
\end{tabularx}
\end{table}

\msubsection{struct}{SystemRecord}
\label{sec:model:SystemRecord}

\begin{table}[ht!]
\begin{tabularx}{\textwidth}{| p{4.25cm} | p{3.5cm} | X |} \hline
\rowcolor{gray!33} Field & Type & Description \\ \hline

address &\pref{Address} & Network address of the system. \\ \hline
authenticationInfo &\pref{String} & X.509 public key of the system. \\ \hline
createdAt & \pref{DateTime} & System instance record was created at this UTC time\-stamp. \\ \hline
id & \pref{Number} & Identifier of the system instance. \\ \hline
metadata &\hyperref[sec:model:Metadata]{Metadata} & Additional information about the system. \\ \hline
port &\pref{PortNumber} & Port of the system. \\ \hline
systemName &\pref{Name} & Name of the system. \\ \hline
updatedAt & \pref{DateTime} & System instance record was modified at this UTC time\-stamp. \\ \hline
\end{tabularx}
\end{table}

\msubsection{struct}{Metadata}
\label{sec:model:Metadata}

An \pref{Object} which maps \pref{String} key-value pairs.

\subsubsection{Primitives}
\label{sec:model:primitives}

As all messages are encoded using the JSON format \cite{bray2014json}, the following primitive constructs, part of that standard, become available.
Note that the official standard is defined in terms of parsing rules, while this list only concerns syntactic information. 

\begin{table}[ht!]
\begin{tabularx}{\textwidth}{| p{3cm} | X |} \hline
\rowcolor{gray!33} JSON Type & Description \\ \hline
\pdef{Value}                 & Any out of \pref{Object}, \pref{Array}, \pref{String}, \pref{Number}, \pref{Boolean} or \pref{Null}. \\ \hline
\pdef{Object}$<$A$>$         & An unordered collection of $[$\pref{String}: \pref{Value}$]$ pairs, where each \pref{Value} conforms to type A. \\ \hline
\pdef{Array}$<$A$>$          & An ordered collection of \pref{Value} elements, where each element conforms to type A. \\ \hline
\pdef{String}                & An arbitrary UTF-8 string. \\ \hline
\pdef{Number}                & Any IEEE 754 binary64 floating point number \cite{cowlishaw2019floating}, except for \textit{+Inf}, \textit{-Inf} and \textit{NaN}. \\ \hline
\pdef{Boolean}               & One out of \texttt{true} or \texttt{false}. \\ \hline
\pdef{Null}                  & Must be \texttt{null}. \\ \hline
\end{tabularx}
\end{table}

With these primitives now available, we proceed to define all the types specified in the \textbf{pull-systems} SD document without a direct equivalent among the JSON types.
Concretely, we define the \textbf{pull-systems} SD primitives either as \textit{aliases} or \textit{structs}.
An \textit{alias} is a renaming of an existing type, but with some further details about how it is intended to be used.
Structs are described in the beginning of the parent section.
The types are listed by name in alphabetical order.

\subsubsection{alias \pdef{Address} = \pref{String}}

A string representation of a network address. An address can be a version 4 IP address (RFC 791), a version 6 IP address (RFC 2460) or a DNS name (RFC 1034).

\subsubsection{alias \pdef{DateTime} = \pref{String}}

Pinpoints a moment in time in the format of ISO8601 standard "yyyy-mm-ddThh:mm:ss", where "yyy" denotes year (4 digits), "mm" denotes month starting from 01, "dd" denotes day starting from 01, "T" is the separator between date and time part, "hh" denotes hour in the 24-hour format (00-23), "MM" denotes minute (00-59), "SS" denotes second (00-59). " " is used as separator between the date and the time.
An example of a valid date/time string is "2020-12-05T12:00:00"

\subsubsection{alias \pdef{Direction} = \pref{String}}

Sorting direction string could be only \texttt{ASC} or \texttt{DESC} . 

\subsubsection{alias \pdef{List}$<$A$>$ = \pref{Array}$<$A$>$}
There is no difference.

\subsubsection{alias \pdef{Name} = \pref{String}}

A \pref{String} indentifier that is intended to be both human and machine-readable.

\subsubsection{alias \pdef{PortNumber} = \pref{Number}}

Decimal \pref{Number} in the range of 0-65535.

\newpage

\bibliographystyle{IEEEtran}
\bibliography{bibliography}

\newpage

\section{Revision History}
\subsection{Amendments}

\noindent\begin{tabularx}{\textwidth}{| p{1cm} | p{3cm} | p{2cm} | X | p{4cm} |} \hline
\rowcolor{gray!33} No. & Date & Version & Subject of Amendments & Author \\ \hline

1 & YYYY-MM-DD & \arrowversion & & Xxx Yyy \\ \hline

\end{tabularx}

\subsection{Quality Assurance}

\noindent\begin{tabularx}{\textwidth}{| p{1cm} | p{3cm} | p{2cm} | X |} \hline
\rowcolor{gray!33} No. & Date & Version & Approved by \\ \hline

1 & YYYY-MM-DD & \arrowversion & Xxx Yyy \\ \hline

\end{tabularx}

\end{document}