\documentclass[a4paper]{arrowhead}

\usepackage[yyyymmdd]{datetime}
\usepackage{etoolbox}
\usepackage[utf8]{inputenc}
\usepackage{multirow}

\renewcommand{\dateseparator}{-}

\setlength{\parskip}{1em}

\newcommand{\fparam}[1]{\textit{\textcolor{ArrowheadBlue}{#1}}}

%% Special references
\newcommand{\fref}[1]{{\textcolor{ArrowheadBlue}{\hyperref[sec:functions:#1]{#1}}}}
\newcommand{\mref}[1]{{\textcolor{ArrowheadPurple}{\hyperref[sec:model:#1]{#1}}}}
\newcommand{\pdef}[1]{{\textcolor{ArrowheadGrey}{#1 \label{sec:model:primitives:#1} \label{sec:model:primitives:#1s}}}}
\newcommand{\pref}[1]{{\textcolor{ArrowheadGrey}{\hyperref[sec:model:primitives:#1]{#1}}}}

\newrobustcmd\fsubsection[5]{
  \addtocounter{subsection}{1}
  \addcontentsline{toc}{subsection}{\protect\numberline{\thesubsection}operation \textcolor{ArrowheadBlue}{#1}}
  \renewcommand*{\do}[1]{\rref{##1},\ }
  \subsection*{
    \thesubsection\quad
    #2 \textcolor{ArrowheadPurple}{#3} \\
    \small
    \hspace*{0.075\textwidth}\begin{minipage}{0.1\textwidth}
      \vspace*{1mm}
      Operation: \\
      \notblank{#4}{Input: \\}{}
      \notblank{#5}{Output: \\}{}
    \end{minipage}
    \begin{minipage}{0.825\textwidth}
      \vspace*{1mm}
      \textcolor{ArrowheadBlue}{#1} \\
      \notblank{#4}{\mref{#4} \\}{}
      \notblank{#5}{\mref{#5} \\}{}
    \end{minipage}
  }
  \label{sec:functions:#1}
}
\newrobustcmd\msubsection[2]{
  \addtocounter{subsection}{1}
  \addcontentsline{toc}{subsection}{\protect\numberline{\thesubsection}#1 \textcolor{ArrowheadPurple}{#2}}
  \subsection*{\thesubsection\quad#1 \textcolor{ArrowheadPurple}{#2}}
  \label{sec:model:#2} \label{sec:model:#2s}
}
\newrobustcmd\msubsubsection[3]{
  \addtocounter{subsubsection}{1}
  \addcontentsline{toc}{subsubsection}{\protect\numberline{\thesubsubsection}#1 \textcolor{ArrowheadPurple}{#2}}
  \subsubsection*{\thesubsubsection\quad#1 \textcolor{ArrowheadPurple}{#2}}
  \label{sec:model:#2} \label{sec:model:#2s}
}
%%

\begin{document}

%% Arrowhead Document Properties
\ArrowheadTitle{service-unregister HTTP/TLS/JSON} %e.g. ServiceDiscovery HTTP/TLS/JSON
\ArrowheadServiceID{service-unregister} % e.g. register
\ArrowheadType{Interface Design Description}
\ArrowheadTypeShort{IDD}
\ArrowheadVersion{4.4.0}
\ArrowheadDate{\today}
\ArrowheadAuthor{Tamás Bordi} % e.g Szvetlin Tanyi}
\ArrowheadStatus{RELEASE}
\ArrowheadContact{tbordi@aitia.ai} % jerker.delsing@arrowhead.eu
\ArrowheadFooter{\href{www.arrowhead.eu}{www.arrowhead.eu}}
\ArrowheadSetup
%%

%% Front Page
\begin{center}
  \vspace*{1cm}
  \huge{\arrowtitle}

  \vspace*{0.2cm}
  \LARGE{\arrowtype}
  \vspace*{1cm}
\end{center}

%  \Large{Service ID: \textit{"\arrowid"}}
  \vspace*{\fill}

  % Front Page Image
  %\includegraphics{figures/TODO}

  \vspace*{1cm}
  \vspace*{\fill}

  % Front Page Abstract
  \begin{abstract}
    This document describes a HTTP protocol with TLS payload
    security and JSON payload encoding variant of the \textbf{service-unregister} service.
  \end{abstract}
  \vspace*{1cm}

\newpage

%% Table of Contents
\tableofcontents
\newpage
%%

\section{Overview}
\label{sec:overview}

This document describes the \textbf{service-unregister} service interface,
which is enables autonomous service unregistration. It's implemented using protocol, encoding as stated in the following table:

\begin{table}[ht!]
  \centering
  \begin{tabular}{|l|l|l|l|}
    \rowcolor{gray!33} Profile ype & Type & Version \\ \hline
    Transfer protocol & HTTP & 1.1 \\ \hline
    Data encryption & TLS & 1.3 \\ \hline
    Encoding & URL & RFC 1738 \\ \hline
    Compression & N/A & - \\ \hline
  \end{tabular}
  \caption{Communication and sematics details used for the \textbf{service-unregister}
    service interface}
  \label{tab:comunication_semantics_profile}
\end{table}

This document provides the Interface Design Description IDD to the \textit{service-unregister -- Service Description} document.
For further details about how this service is meant to be used, please consult that document.

The rest of this document describes how to realize the service-unregister service HTTP/TLS/JSON interface in details.

\newpage

\section{Interface Description}
\label{sec:functions}

The service responses with the status code \texttt{200
  Ok} if called successfully. The error codes are, \texttt{400
  Bad Request} if request is malformed, \texttt{401 Unauthorized} if
improper client side certificate is provided, \texttt{500 Internal
  Server Error} if Service Registry is unavailable.

\begin{lstlisting}[language=http,label={lst:unregister},caption={A \fref{service-unregister} invocation.}]
DELETE /serviceregistry/unregister?service_definition={serviceDefinition}&system_name={providerName}&address={providerAddress}&port={providerPort}&service_uri={serviceUri} HTTP/1.1
\end{lstlisting}

\newpage

\section{Data Models}
\label{sec:model}

Here, all data objects that can be part of the service calls associated with this service are listed in alphabetic order.
Note that each subsection, which describes one type of object, begins with the \textit{struct} keyword, which is meant to denote a JSON \pref{Object} that must contain certain fields, or names, with values conforming to explicitly named types.
As a complement to the primary types defined in this section, there is also a list of secondary types in Section \ref{sec:model:primitives}, which are used to represent things like hashes, identifiers and texts.

\msubsection{struct}{QueryParams}

This structure is used to unregister a service from Service Registry.

\begin{table}[ht!]
\begin{tabularx}{\textwidth}{| p{3cm} | p{3cm} | p{2cm} | X |} \hline
\rowcolor{gray!33} Field & Type & Mandatory & Description \\ \hline
serviceDefinition &\pref{Name} & yes & Identifier of the service. \\ \hline
providerName & \pref{Name} & yes & Identifier of the provider system. \\ \hline
providerAddress & \pref{Address} & no & Network address. \\ \hline
providerPort &\pref{PortNumber} & yes & Port of the system. \\ \hline
serviceUri &\pref{URI} & yes & URI of the service. \\ \hline
\end{tabularx}
\end{table}

\newpage

\subsection{Primitives}
\label{sec:model:primitives}

\begin{table}[ht!]
\begin{tabularx}{\textwidth}{| p{3cm} | X |} \hline
\rowcolor{gray!33} Type & Description \\ \hline
\pdef{Address}          & A string representation of the address \\ \hline
\pdef{Name}             & A string identifier that is intended to be both human and machine-readable. \\ \hline
\pdef{Number}           & Decimal number \\ \hline
\pdef{PortNumber}       & Decimal number in the range of 0-65535 \\ \hline
\pdef{String}           & An arbitrary UTF-8 string. \\ \hline
\end{tabularx}
\end{table}

With these primitives now available, we proceed to define all the types specified in the \textbf{service-uregister} SD document without a direct equivalent among the JSON types.
Concretely, we define the \textbf{service-uregister} SD primitives either as \textit{aliases} or \textit{structs}.
An \textit{alias} is a renaming of an existing type, but with some further details about how it is intended to be used.
Structs are described in the beginning of the parent section.
The types are listed by name in alphabetical order.

\subsubsection{alias \pdef{Name} = \pref{String}}

A \pref{String} indentifier that is intended to be both human and machine-readable.

\subsubsection{alias \pdef{PortNumber} = \pref{Number}}

Decimal \pref{Number} in the range of 0-65535.


\subsubsection{alias \pdef{URI} = \pref{String}}

A \pref{String} that represents the URL subpath where the offered service is reachable, starting with a slash ("/"). An example of a valid URI is "/temperature".

\newpage

\bibliographystyle{IEEEtran}
\bibliography{bibliography}

\newpage

\section{Revision History}
\subsection{Amendments}

\noindent\begin{tabularx}{\textwidth}{| p{1cm} | p{3cm} | p{2cm} | X | p{4cm} |} \hline
\rowcolor{gray!33} No. & Date & Version & Subject of Amendments & Author \\ \hline

1 & YYYY-MM-DD & \arrowversion & & Xxx Yyy \\ \hline

\end{tabularx}

\subsection{Quality Assurance}

\noindent\begin{tabularx}{\textwidth}{| p{1cm} | p{3cm} | p{2cm} | X |} \hline
\rowcolor{gray!33} No. & Date & Version & Approved by \\ \hline

1 & YYYY-MM-DD & \arrowversion & Xxx Yyy \\ \hline

\end{tabularx}

\end{document}