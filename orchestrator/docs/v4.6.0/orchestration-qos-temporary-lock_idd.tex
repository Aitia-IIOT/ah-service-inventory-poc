\documentclass[a4paper]{arrowhead}

\usepackage[yyyymmdd]{datetime}
\usepackage{etoolbox}
\usepackage[utf8]{inputenc}
\usepackage{multirow}

\renewcommand{\dateseparator}{-}

\setlength{\parskip}{1em}

\newcommand{\fparam}[1]{\textit{\textcolor{ArrowheadBlue}{#1}}}

%% Special references
\newcommand{\fref}[1]{{\textcolor{ArrowheadBlue}{\hyperref[sec:functions:#1]{#1}}}}
\newcommand{\mref}[1]{{\textcolor{ArrowheadPurple}{\hyperref[sec:model:#1]{#1}}}}
\newcommand{\pdef}[1]{{\textcolor{ArrowheadGrey}{#1 \label{sec:model:primitives:#1} \label{sec:model:primitives:#1s}}}}
\newcommand{\pref}[1]{{\textcolor{ArrowheadGrey}{\hyperref[sec:model:primitives:#1]{#1}}}}

\newrobustcmd\fsubsection[5]{
  \addtocounter{subsection}{1}
  \addcontentsline{toc}{subsection}{\protect\numberline{\thesubsection}operation \textcolor{ArrowheadBlue}{#1}}
  \renewcommand*{\do}[1]{\rref{##1},\ }
  \subsection*{
    \thesubsection\quad
    #2 \textcolor{ArrowheadPurple}{#3} \\
    \small
    \hspace*{0.075\textwidth}\begin{minipage}{0.1\textwidth}
      \vspace*{1mm}
      Operation: \\
      \notblank{#4}{Input: \\}{}
      \notblank{#5}{Output: \\}{}
    \end{minipage}
    \begin{minipage}{0.825\textwidth}
      \vspace*{1mm}
      \textcolor{ArrowheadBlue}{#1} \\
      \notblank{#4}{\mref{#4} \\}{}
      \notblank{#5}{\mref{#5} \\}{}
    \end{minipage}
  }
  \label{sec:functions:#1}
}
\newrobustcmd\msubsection[2]{
  \addtocounter{subsection}{1}
  \addcontentsline{toc}{subsection}{\protect\numberline{\thesubsection}#1 \textcolor{ArrowheadPurple}{#2}}
  \subsection*{\thesubsection\quad#1 \textcolor{ArrowheadPurple}{#2}}
  \label{sec:model:#2} \label{sec:model:#2s}
}
\newrobustcmd\msubsubsection[3]{
  \addtocounter{subsubsection}{1}
  \addcontentsline{toc}{subsubsection}{\protect\numberline{\thesubsubsection}#1 \textcolor{ArrowheadPurple}{#2}}
  \subsubsection*{\thesubsubsection\quad#1 \textcolor{ArrowheadPurple}{#2}}
  \label{sec:model:#2} \label{sec:model:#2s}
}
%%

\begin{document}

%% Arrowhead Document Properties
\ArrowheadTitle{orchestration-qos-temporary-lock HTTP/TLS/JSON} %e.g. ServiceDiscovery HTTP/TLS/JSON
\ArrowheadServiceID{orchestration-qos-temporary-lock} % e.g. register
\ArrowheadType{Interface Design Description}
\ArrowheadTypeShort{IDD}
\ArrowheadVersion{4.6.0}
\ArrowheadDate{\today}
\ArrowheadAuthor{Rajmund Bocsi} % e.g Szvetlin Tanyi}
\ArrowheadStatus{RELEASE}
\ArrowheadContact{rbocsi@aitia.ai} % jerker.delsing@arrowhead.eu
\ArrowheadFooter{\href{www.arrowhead.eu}{www.arrowhead.eu}}
\ArrowheadSetup
%%

%% Front Page
\begin{center}
  \vspace*{1cm}
  \huge{\arrowtitle}

  \vspace*{0.2cm}
  \LARGE{\arrowtype}
  \vspace*{1cm}
\end{center}

%  \Large{Service ID: \textit{"\arrowid"}}
  \vspace*{\fill}

  % Front Page Image
  %\includegraphics{figures/TODO}

  \vspace*{1cm}
  \vspace*{\fill}

  % Front Page Abstract
  \begin{abstract}
    This document describes a HTTP protocol with TLS payload
    security and JSON payload encoding variant of the \textbf{orchestration-qos-temporary-lock} service.
  \end{abstract}
  \vspace*{1cm}

\newpage

%% Table of Contents
\tableofcontents
\newpage
%%

\section{Overview}
\label{sec:overview}

This document describes the \textbf{orchestration-qos-temporary-lock} service interface, that enables to the requester system to exclude some providers from orchestration temporarily. It's implemented using protocol, encoding as stated in the following table:

\begin{table}[ht!]
  \centering
  \begin{tabular}{|l|l|l|l|}
    \rowcolor{gray!33} Profile type & Type & Version \\ \hline
    Transfer protocol & HTTP & 1.1 \\ \hline
    Data encryption & TLS & 1.3 \\ \hline
    Encoding & JSON & RFC 8259 \cite{rfc8259} \\ \hline
    Compression & N/A & - \\ \hline
  \end{tabular}
  \caption{Communication and semantics details used for the \textbf{orchestration-qos-temporary-lock} service interface}
  \label{tab:comunication_semantics_profile}
\end{table}

This document provides the Interface Design Description IDD to the \textit{orchestration-qos-temporary-lock -- Service Description} document.
For further details about how this service is meant to be used, please consult that document.

The rest of this document describes how to realize the \textbf{orchestration-qos-temporary-lock} service HTTP/ TLS/JSON interface in details.

\newpage

\section{Interface Description}
\label{sec:functions}

The service responses with the status code \texttt{200 Ok} if called successfully. The error codes are, \texttt{400 Bad Request} if request is malformed, \texttt{401 Unauthorized} if improper client side certificate is provided, \texttt{500 Internal Server Error} if Orchestrator is unavailable.

\begin{lstlisting}[language=http,label={lst:orchestration-qos-temporary-lock},caption={An \fref{orchestration-qos-temporary-lock} invocation.}]
POST /orchestrator/qos_temporary_lock HTTP/1.1

{
  "requester": {
    "systemName": "string",
    "address": "string",
    "port": 0,
    "authenticationInfo": "string"
  },
  "orList":  [
    {
      "provider": {
        "id": 1,
        "systemName": "string",
        "address": "string",
        "port": 0,
        "authenticationInfo": "string",
        "metadata": {
          "additionalProp1": "string",
          "additionalProp2": "string",
          "additionalProp3": "string"
        },  
        "createdAt": "string",
        "updatedAt": "string"
      },
      "service": {
        "id": 2,
        "serviceDefinition": "string",
        "createdAt": "string",
        "updatedAt": "string"
      },
      "serviceUri": "string",
      "secure": "TOKEN",
      "metadata": {
        "additionalProp1": "string",
        "additionalProp2": "string",
        "additionalProp3": "string"
      },  
      "interfaces": [
        {
          "id": 0,
          "createdAt": "string",
          "interfaceName": "string",
          "updatedAt": "string"
        }
      ],
      "version": 0,
      "authorizationTokens": {
        "interfaceName1": "token1",
        "interfaceName2": "token2"
      },
      "warnings": [
        "TTL_UNKNOWN"
      ]
    }
  ]
}
\end{lstlisting}

\clearpage

\begin{lstlisting}[language=http,label={lst:orchestration-qos-temporary-lock_response},caption={An \fref{orchestration-qos-temporary-lock} response.}]
{
  "response": [
    {
      "provider": {
        "id": 1,
        "systemName": "string",
        "address": "string",
        "port": 0,
        "authenticationInfo": "string",
        "metadata": {
          "additionalProp1": "string",
          "additionalProp2": "string",
          "additionalProp3": "string"
        },  
        "createdAt": "string",
        "updatedAt": "string"
      },
      "service": {
        "id": 2,
        "serviceDefinition": "string",
        "createdAt": "string",
        "updatedAt": "string"
      },
      "serviceUri": "string",
      "secure": "TOKEN",
      "metadata": {
        "additionalProp1": "string",
        "additionalProp2": "string",
        "additionalProp3": "string"
      },  
      "interfaces": [
        {
          "id": 0,
          "createdAt": "string",
          "interfaceName": "string",
          "updatedAt": "string"
        }
      ],
      "version": 0,
      "authorizationTokens": {
        "interfaceName1": "token1",
        "interfaceName2": "token2"
      },
      "warnings": [
        "TTL_UNKNOWN"
      ]
    }
  ]
}
\end{lstlisting}

\newpage

\section{Data Models}
\label{sec:model}

Here, all data objects that can be part of the service calls associated with this service are listed in alphabetic order.
Note that each subsection, which describes one type of object, begins with the \textit{struct} keyword, which is meant to denote a JSON \pref{Object} that must contain certain fields, or names, with values conforming to explicitly named types.
As a complement to the primary types defined in this section, there is also a list of secondary types in Section \ref{sec:model:primitives}, which are used to represent things like hashes, identifiers and texts.

\msubsection{struct}{QoSLockRequest}
\label{sec:model:QoSLockRequest}

\begin{table}[ht!]
\begin{tabularx}{\textwidth}{| p{4cm} | p{4cm} | p{2cm} | X |} \hline
\rowcolor{gray!33} Field & Type & Mandatory & Description \\ \hline
requester & \hyperref[sec:model:System]{System} & yes & The consumer system on whose behalf the lock request is sent. \\ \hline
orList & \pref{List}$<$\hyperref[sec:model:OrchestrationResult]{OrchestrationResult}$>$ & no & A result list of an orchestration request. The providers in the list are the subjects of the lock. \\ \hline

\end{tabularx}
\end{table}

\msubsection{struct}{System}
\label{sec:model:System}

\begin{table}[ht!]
\begin{tabularx}{\textwidth}{| p{4cm} | p{4cm} | p{2cm} | X |} \hline
\rowcolor{gray!33} Field & Type & Mandatory & Description \\ \hline

address &\pref{Address} & yes & Network address of the system. \\ \hline
authenticationInfo &\pref{String} & no & X.509 public key of the system. \\ \hline
metadata &\hyperref[sec:model:Metadata]{Metadata} & no & Additional information about the system. \\ \hline
port &\pref{PortNumber} & yes & Port of the system. \\ \hline
systemName &\pref{Name} & yes & Name of the system. \\ \hline
\end{tabularx}
\end{table}

\msubsection{struct}{Metadata}
\label{sec:model:Metadata}

An \pref{Object} which maps \pref{String} key-value pairs.

\clearpage

\msubsection{struct}{OrchestrationResult}
\label{sec:model:OrchestrationResult}

\begin{table}[ht!]
\begin{tabularx}{\textwidth}{| p{3.5cm} | p{4.6cm} | p{2cm} | X |} \hline
\rowcolor{gray!33} Field & Type & Mandatory & Description \\ \hline
authorizationTokens & \hyperref[sec:model:Metadata]{Metadata} & no & Tokens to use the service instance (one for every supported interface). Only filled if the security type is \texttt{TOKEN}. \\ \hline
interfaces & \pref{List}$<$\hyperref[sec:model:ServiceInterfaceRecord]{ServiceInterfaceRecord}$>$ & no & List of interfaces the service instance supports. \\ \hline
metadata & \hyperref[sec:model:Metadata]{Metadata} & no & Service instance metadata. \\ \hline
provider & \hyperref[sec:model:SystemRecord]{SystemRecord} & yes & Descriptor of the provider system record. \\ \hline
secure & \pref{SecureType} & no & Type of security the service instance uses. \\ \hline
service & \hyperref[sec:model:ServiceDefinitionRecord]{ServiceDefinitionRecord} & yes & Descriptor of the service definition record. \\ \hline
serviceUri & \pref{String} & no & Path of the service on the provider. \\ \hline
version & \pref{Version} & no & Version of the service instance. \\ \hline
warnings & \pref{List}$<$\pref{OrchestratorWarning}$>$ & no & List of warnings about the provider and/or its service instance. \\ \hline

\end{tabularx}
\end{table}

\msubsection{struct}{ServiceInterfaceRecord}
\label{sec:model:ServiceInterfaceRecord}

\begin{table}[ht!]
\begin{tabularx}{\textwidth}{| p{4cm} | p{4cm} | p{2cm} | X |} \hline
\rowcolor{gray!33} Field & Type & Mandatory & Description \\ \hline
createdAt & \pref{DateTime} & no & Interface instance record was created at this UTC time\-stamp. \\ \hline
id & \pref{Number} & no & Identifier of the interface instance. \\ \hline
interfaceName &\pref{Interface} & no & Specified name of the interface. \\ \hline
updatedAt & \pref{DateTime} & no & Interface instance record was modified at this UTC time\-stamp. \\ \hline
\end{tabularx}
\end{table}

\clearpage

\msubsection{struct}{SystemRecord}
\label{sec:model:SystemRecord}

\begin{table}[ht!]
\begin{tabularx}{\textwidth}{| p{4cm} | p{4cm} | p{2cm} | X |} \hline
\rowcolor{gray!33} Field & Type & Mandatory & Description \\ \hline

address &\pref{Address} & no & Network address of the system. \\ \hline
authenticationInfo &\pref{String} & no & X.509 public key of the system. \\ \hline
createdAt & \pref{DateTime} & no & System instance record was created at this UTC time\-stamp. \\ \hline
id & \pref{Number} & yes & Identifier of the system instance. \\ \hline
metadata &\hyperref[sec:model:Metadata]{Metadata} & no & Additional information about the system. \\ \hline
port &\pref{PortNumber} & no & Port of the system. \\ \hline
systemName &\pref{Name} & no & Name of the system. \\ \hline
updatedAt & \pref{DateTime} & no & System instance record was modified at this UTC time\-stamp. \\ \hline
\end{tabularx}
\end{table}

\msubsection{struct}{ServiceDefinitionRecord}
\label{sec:model:ServiceDefinitionRecord}

\begin{table}[ht!]
\begin{tabularx}{\textwidth}{| p{4cm} | p{4cm} | p{2cm} | X |} \hline
\rowcolor{gray!33} Field & Type & Mandatory & Description \\ \hline
createdAt & \pref{DateTime} & no & Service definition instance record was created at this UTC time\-stamp. \\ \hline
id & \pref{Number} & yes & Identifier of the service definition instance. \\ \hline
serviceDefinition &\pref{Name} & no  & Name of the service definition. \\ \hline
updatedAt & \pref{DateTime} & no & Service definition instance record was modified at this UTC time\-stamp. \\ \hline
\end{tabularx}
\end{table}

\msubsection{struct}{QoSLockResponse}
\label{sec:model:QoSLockResponse}

\begin{table}[ht!]
\begin{tabularx}{\textwidth}{| p{4cm} | p{4cm} | X |} \hline
\rowcolor{gray!33} Field & Type & Description \\ \hline
response & \pref{List}$<$\hyperref[sec:model:OrchestrationResult]{OrchestrationResult}$>$ & Orchestration results belong to providers that are successfully locked. \\ \hline
\end{tabularx}
\end{table}

\clearpage

\subsection{Primitives}
\label{sec:model:primitives}

As all messages are encoded using the JSON format \cite{bray2014json}, the following primitive constructs, part of that standard, become available.
Note that the official standard is defined in terms of parsing rules, while this list only concerns syntactic information.
Furthermore, the \pref{Object} and \pref{Array} types are given optional generic type parameters, which are used in this document to signify when pair values or elements are expected to conform to certain types. 

\begin{table}[ht!]
\begin{tabularx}{\textwidth}{| p{3cm} | X |} \hline
\rowcolor{gray!33} JSON Type & Description \\ \hline
\pdef{Value}                 & Any out of \pref{Object}, \pref{Array}, \pref{String}, \pref{Number}, \pref{Boolean} or \pref{Null}. \\ \hline
\pdef{Object}$<$A$>$         & An unordered collection of $[$\pref{String}: \pref{Value}$]$ pairs, where each \pref{Value} conforms to type A. \\ \hline
\pdef{Array}$<$A$>$          & An ordered collection of \pref{Value} elements, where each element conforms to type A. \\ \hline
\pdef{String}                & An arbitrary UTF-8 string. \\ \hline
\pdef{Number}                & Any IEEE 754 binary64 floating point number \cite{cowlishaw2019floating}, except for \textit{+Inf}, \textit{-Inf} and \textit{NaN}. \\ \hline
\pdef{Boolean}               & One out of \texttt{true} or \texttt{false}. \\ \hline
\pdef{Null}                  & Must be \texttt{null}. \\ \hline
\end{tabularx}
\end{table}

With these primitives now available, we proceed to define all the types specified in the \textbf{orchestration-qos-temporary-lock} SD document without a direct equivalent among the JSON types.
Concretely, we define the \textbf{orchestration-qos-temporary-lock} SD primitives either as \textit{aliases} or \textit{structs}.
An \textit{alias} is a renaming of an existing type, but with some further details about how it is intended to be used.
Structs are described in the beginning of the parent section.
The types are listed by name in alphabetical order.

\subsubsection{alias \pdef{Address} = \pref{String}}

A string representation of a network address. An address can be a version 4 IP address (RFC 791), a version 6 IP address (RFC 2460) or a DNS name (RFC 1034).

\subsubsection{alias \pdef{DateTime} = \pref{String}}

Pinpoints a moment in time in the format of ISO8601 standard "yyyy-mm-ddThh:mm:ss", where "yyy" denotes year (4 digits), "mm" denotes month starting from 01, "dd" denotes day starting from 01, "T" is the separator between date and time part, "hh" denotes hour in the 24-hour format (00-23), "MM" denotes minute (00-59), "SS" denotes second (00-59). " " is used as separator between the date and the time.
An example of a valid date/time string is "2020-12-05T12:00:00"

\subsubsection{alias \pdef{Interface} = \pref{String}}

A \pref{String} that describes an interface in \textit{Protocol-SecurityType-MimeType} format. \textit{SecurityType} can be SECURE or INSECURE. \textit{Protocol} and \textit{MimeType} can be anything. An example of a valid interface is: "HTTP-SECURE-JSON" or "HTTP-INSECURE-SENML".

\subsubsection{alias \pdef{List}$<$A$>$ = \pref{Array}$<$A$>$}
There is no difference.

\subsubsection{alias \pdef{Name} = \pref{String}}

A \pref{String} identifier that is intended to be both human and machine-readable.

\subsubsection{alias \pdef{PortNumber} = \pref{Number}}

Decimal \pref{Number} in the range of 0-65535.

\subsubsection{alias \pdef{OrchestratorWarning} = \pref{String}}

A String that represents a potentially interesting information about a provider and/or its service instance. Possible values are \textit{FROM\_OTHER\_CLOUD} (if the provider is in an other cloud), \textit{TTL\_EXPIRED} (the provider is no longer accessible), \textit{TTL\_EXPIRING} (the provider will be inaccessible in a matter of minutes), \textit{TTL\_UNKNOWN} (the provider does not specified expiration time), \textit{VIA\_GATEWAY} (the provider is in an other cloud and only accessible via a tunnel provided by the Gateway Core System)

\subsubsection{alias \pdef{SecureType} = \pref{String}}

A \pref{String} that describes an the security type. Possible values are \textit{NOT\_SECURE} or \textit{CERTIFICATE} or \textit{TOKEN}.

\subsubsection{alias \pdef{Version} = \pref{Number}}

A \pref{Number} that represents the version of the service. And example of a valid version is: 1.
\color{black}

\newpage

\bibliographystyle{IEEEtran}
\bibliography{bibliography}

\newpage

\section{Revision History}
\subsection{Amendments}

\noindent\begin{tabularx}{\textwidth}{| p{1cm} | p{3cm} | p{2cm} | X | p{4cm} |} \hline
\rowcolor{gray!33} No. & Date & Version & Subject of Amendments & Author \\ \hline

1 & YYYY-MM-DD & \arrowversion & & Xxx Yyy \\ \hline

\end{tabularx}

\subsection{Quality Assurance}

\noindent\begin{tabularx}{\textwidth}{| p{1cm} | p{3cm} | p{2cm} | X |} \hline
\rowcolor{gray!33} No. & Date & Version & Approved by \\ \hline

1 & YYYY-MM-DD & \arrowversion & Xxx Yyy \\ \hline

\end{tabularx}

\end{document}