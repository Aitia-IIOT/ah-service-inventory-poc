\documentclass[a4paper]{arrowhead}

\usepackage[yyyymmdd]{datetime}
\usepackage{etoolbox}
\usepackage[utf8]{inputenc}
\usepackage{multirow}

\renewcommand{\dateseparator}{-}

\setlength{\parskip}{1em}

\newcommand{\fparam}[1]{\textit{\textcolor{ArrowheadBlue}{#1}}}

%% Special references
\newcommand{\fref}[1]{{\textcolor{ArrowheadBlue}{\hyperref[sec:functions:#1]{#1}}}}
\newcommand{\mref}[1]{{\textcolor{ArrowheadPurple}{\hyperref[sec:model:#1]{#1}}}}
\newcommand{\pdef}[1]{{\textcolor{ArrowheadGrey}{#1 \label{sec:model:primitives:#1} \label{sec:model:primitives:#1s}}}}
\newcommand{\pref}[1]{{\textcolor{ArrowheadGrey}{\hyperref[sec:model:primitives:#1]{#1}}}}

\newrobustcmd\fsubsection[5]{
  \addtocounter{subsection}{1}
  \addcontentsline{toc}{subsection}{\protect\numberline{\thesubsection}operation \textcolor{ArrowheadBlue}{#1}}
  \renewcommand*{\do}[1]{\rref{##1},\ }
  \subsection*{
    \thesubsection\quad
    #2 \textcolor{ArrowheadPurple}{#3} \\
    \small
    \hspace*{0.075\textwidth}\begin{minipage}{0.1\textwidth}
      \vspace*{1mm}
      Operation: \\
      \notblank{#4}{Input: \\}{}
      \notblank{#5}{Output: \\}{}
    \end{minipage}
    \begin{minipage}{0.825\textwidth}
      \vspace*{1mm}
      \textcolor{ArrowheadBlue}{#1} \\
      \notblank{#4}{\mref{#4} \\}{}
      \notblank{#5}{\mref{#5} \\}{}
    \end{minipage}
  }
  \label{sec:functions:#1}
}
\newrobustcmd\msubsection[2]{
  \addtocounter{subsection}{1}
  \addcontentsline{toc}{subsection}{\protect\numberline{\thesubsection}#1 \textcolor{ArrowheadPurple}{#2}}
  \subsection*{\thesubsection\quad#1 \textcolor{ArrowheadPurple}{#2}}
  \label{sec:model:#2} \label{sec:model:#2s}
}
\newrobustcmd\msubsubsection[3]{
  \addtocounter{subsubsection}{1}
  \addcontentsline{toc}{subsubsection}{\protect\numberline{\thesubsubsection}#1 \textcolor{ArrowheadPurple}{#2}}
  \subsubsection*{\thesubsubsection\quad#1 \textcolor{ArrowheadPurple}{#2}}
  \label{sec:model:#2} \label{sec:model:#2s}
}
%%

\begin{document}

%% Arrowhead Document Properties
\ArrowheadTitle{orchestration-qos-enabled HTTP/TLS/TEXT} %e.g. ServiceDiscovery HTTP/TLS/JSON
\ArrowheadServiceID{orchestration-qos-enabled} % e.g. register
\ArrowheadType{Interface Design Description}
\ArrowheadTypeShort{IDD}
\ArrowheadVersion{4.6.0}
\ArrowheadDate{\today}
\ArrowheadAuthor{Rajmund Bocsi} % e.g Szvetlin Tanyi}
\ArrowheadStatus{RELEASE}
\ArrowheadContact{rbocsi@aitia.ai} % jerker.delsing@arrowhead.eu
\ArrowheadFooter{\href{www.arrowhead.eu}{www.arrowhead.eu}}
\ArrowheadSetup
%%

%% Front Page
\begin{center}
  \vspace*{1cm}
  \huge{\arrowtitle}

  \vspace*{0.2cm}
  \LARGE{\arrowtype}
  \vspace*{1cm}
\end{center}

%  \Large{Service ID: \textit{"\arrowid"}}
  \vspace*{\fill}

  % Front Page Image
  %\includegraphics{figures/TODO}

  \vspace*{1cm}
  \vspace*{\fill}

  % Front Page Abstract
  \begin{abstract}
    This document describes a HTTP protocol with TLS payload
    security and TEXT payload encoding variant of the \textbf{orchestration-qos-enabled} service.
  \end{abstract}
  \vspace*{1cm}

\newpage

%% Table of Contents
\tableofcontents
\newpage
%%

\section{Overview}
\label{sec:overview}

This document describes the \textbf{orchestration-qos-enabled} service interface, which tells the requester whether the provider supports Quality-of-Service requirements or not. It's implemented using protocol, encoding as stated in the following table:

\begin{table}[ht!]
  \centering
  \begin{tabular}{|l|l|l|l|}
    \rowcolor{gray!33} Profile type & Type & Version \\ \hline
    Transfer protocol & HTTP & 1.1 \\ \hline
    Data encryption & TLS & 1.3 \\ \hline
    Encoding & TEXT & - \\ \hline
    Compression & N/A & - \\ \hline
  \end{tabular}
  \caption{Communication and semantics details used for the \textbf{orchestration-qos-enabled}
    service interface}
  \label{tab:comunication_semantics_profile}
\end{table}

This document provides the Interface Design Description IDD to the \textit{orchestration-qos-enabled -- Service Description} document.
For further details about how this service is meant to be used, please consult that document.

The rest of this document describes how to realize the \textbf{orchestration-qos-enabled} service HTTP/TLS/TEXT interface in details.

\newpage

\section{Interface Description}
\label{sec:functions}

The service responses with the status code \texttt{200  Ok} if called successfully. The error codes are \texttt{401 Unauthorized} if improper client side certificate is provided and \texttt{500 Internal Server Error} if the system is unavailable.

\begin{lstlisting}[language=http,label={lst:orchestration-qos-enabled},caption={An \fref{orchestration-qos-enabled} invocation.}]
GET /orchestrator/qos_enabled HTTP/1.1
\end{lstlisting}

\begin{lstlisting}[language=http,label={lst:orchestration-qos-enabled_response},caption={An \fref{orchestration-qos-enabled} response.}]
true
\end{lstlisting}

\newpage

\section{Data Models}
\label{sec:model}

Here, all data objects that can be part of the service calls associated with this service are listed in alphabetic order.

\msubsection{text}{QoSEnabledResponse}

Simple text message with the value of "true" or "false" depending on whether the provider supports Quality-of-Service requirements or not.

\newpage

\bibliographystyle{IEEEtran}
\bibliography{bibliography}

\newpage

\section{Revision History}
\subsection{Amendments}

\noindent\begin{tabularx}{\textwidth}{| p{1cm} | p{3cm} | p{2cm} | X | p{4cm} |} \hline
\rowcolor{gray!33} No. & Date & Version & Subject of Amendments & Author \\ \hline

1 & YYYY-MM-DD & \arrowversion & & Xxx Yyy \\ \hline

\end{tabularx}

\subsection{Quality Assurance}

\noindent\begin{tabularx}{\textwidth}{| p{1cm} | p{3cm} | p{2cm} | X |} \hline
\rowcolor{gray!33} No. & Date & Version & Approved by \\ \hline

1 & YYYY-MM-DD & \arrowversion & Xxx Yyy \\ \hline

\end{tabularx}

\end{document}